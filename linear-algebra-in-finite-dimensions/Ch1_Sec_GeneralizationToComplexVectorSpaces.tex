\section{Generalization to Complex Vector Spaces}
We now make the generalization to complex vector spaces as promised.  This is really quite simple because the only thing that needs modification is the inner product.

So far we have dealt with the vector spaces \textbf{R}$^n$, the spaces of n-tuples of real numbers.  Consider now the set of n-tuples of complex numbers denoted \textbf{C}$^n$.  It is very easy to check that this is a vector space, addition and scalar multiplication are defined exactly the same way that they were in the case of \textbf{R}$^n$.  It is only when we try to invent an inner product for \textbf{C}$^n$ that we find trouble.  Consider for example the vector $\ket{v} = (0,e^{i\frac{\pi}{4}})$ in \textbf{C}$^2$.  If we compute the norm squared of $\ket{v}$ by simply squaring components and adding we get $||v||^2 = i$.  This is bad because it violates property \textit{2} of our definition of inner product which says that the norm squared of a vector should be a non-negative real number.  The reason that we want our vectors to have real norms is that we use the norm as a measure of a vector's size.  If a vector has a complex norm you can't meaningfully compare it to another vector; for example, you can't tell whether a vector with a norm of $i$ is bigger than a vector with a norm of $1$.  So we see that we have to modify our inner product for \textbf{C}$^n$.  To fix the problem of finding vectors with complex norms we need to garuntee that when we compute $\braket{v}{v}$ each term is a positive real number.  We do this simply by defining the inner product by the equation
\begin{displaymath} \braket{v}{w} = v_1^*w_1 + v_2^*w_2 + \cdots + v_n^*w_n = \sum_{i=1}^n v_i^*w_i. \end{displaymath}
where $^*$ means complex conjugate.  This definition ensures that all vectors will have a non-negative real norm as can be seen by the following computation,
\begin{displaymath} \norm{v}^2 = \braket{v}{v} = v_1^*v_1 + v_2^*v_2 + \cdots + v_n^*v_n = \sum_{i=1}^n |v_i|^2. \end{displaymath}
We should point out that property \textit{1} of the definition of the inner product must be altered when working in complex spaces.  It is clear from the form of the inner product in complex spaces that the we no longer have $\braket{v}{w} = \braket{w}{v}$.  We instead have $\braket{v}{w} = \braket{w}{v}^*$.

In complex spaces we call the dual of an operator $T$ the \textbf{Hermitian conjugate} of $T$.  It is still denoted by the symbol $T^{\dag}$ and defined by the equation
\begin{displaymath} \braket{u}{Tv} = \braket{T^{\dag}u}{v}. \end{displaymath}
In complex spaces an operator $T$ satisfying the equation $T^{\dag} = T$ is said to be \textbf{Hermitian}.  You can show that if an operator has a matrix $[T]$ then the matrix of the Hermitian conjugate is $[T]^{\top *}$, which means the complex conjugate of the transpose.  Rather than write $[T]^{\top *}$ which is cumbersome it is usual to write $[T]^{\dag}$ for the Hermitian conjugate matrix.  Note that $[T]^{\dag}$ is not the same thing as $[T]^{\top}$, they differ by complex conjugation.  This difference must be kept in mind when using the formulas for changing vector and matrix elements from one basis to another. 

\begin{flushleft}\textbf{Exercises}\end{flushleft}