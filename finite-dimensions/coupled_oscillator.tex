\levelstay{The Coupled Oscillator}
Consider the system shown in Figure \ref{Fig:CoupledMasses} consisting of two boxes of equal mass $m$ connected by three springs of equal spring constant $k$.
Suppose we want to know, given initial positions and velocities of the boxes at time $t_{0}$, what are their positions and velocities at another time $t$?
To answer this question,  we just have to solve Newton's equation of motion for each box, but this is not so easy!
The equations of motion for box \#1 is
\begin{align*}
F &= m \ddot{x} \\
F_\text{spring 1} + F_\text{spring 2} &= m \ddot{x} \\
-k x_1 + k (x_2 - x_1) &= m \ddot{x} \\
\ddot{x}_1 &= \omega_0^2 (-2 x_1 + x_2) \, , \\
\end{align*}
where $\omega_0^2 \equiv k/m$.\footnote{We use the symbol $\omega_0$ here because, as we'll see later, $\sqrt{k/m}$ is the oscillation frequency of each box if they were uncoupled.}
Similarly, the equation of motion for box \#2 is
\begin{equation*}
\ddot{x}_2 = \omega_0^2 (-2 x_2 + x_1) \, ,
\end{equation*}
giving us the system of equations
\begin{equation} \label{eq:Newton}
\ddot{x}_{1} = \omega_{0}^{2} (-2x_{1} + x_{2}) \qquad
\ddot{x}_{2} = \omega_{0}^{2} (-2x_{2} + x_{1}) \, .
\end{equation}
This is a set of two second order differential equations which are \textit{coupled}, meaning that each equation has dependance on both $x_1$ and $x_2$.
If you haven't taken a course in linear algebra or differential equations, it looks very difficult to solve.
Give it a shot: try to solve for $x_{1}(t)$ and $x_{2}(t)$.

\quickfig
{\columnwidth}
{boxes.pdf}
{Coupled mass system.
Each spring has spring constant $k$ and each box has mass $m$.
The walls are fixed in place.}
{Fig:CoupledMasses}

\leveldown{Exercises}

\begin{itemize}\item[1)] Verify that (\ref{eq:Newton}) correctly describes the motion of the two coupled masses.\item[2)] Try to solve for $x_{1}(t)$ and $x_{2}(t)$.\end{itemize}

