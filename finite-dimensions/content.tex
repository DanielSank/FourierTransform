\levelstay{Linear algebra in finite dimensions}

This chapter discusses linear algebra in vector spaces that have finite dimension.
These are the vector spaces that are probably most familiar to you, such as the space $\reals^{n}$ studied in introductory linear algebra courses.
Our emphasis will be on understanding vectors as objects independent of the particular choice of basis that is used to represent them.
We will then look at linear transformations of vectors and see how those linear transformations are represented by matrices.
We will see that the matrix representation of a linear transformation is not unique; it depends on the choice of basis used to represent the vector space.
To tie the ideas together and to show why these things are of practical interest, we will use all of our mathematical machinery on a certain physical system that appears mathematically complicated: the coupled oscillator.
By the end of this chapter we will be able to solve the equations of motion for the coupled oscillators in an elegant and very easy way.
The beginnings and ends of sections discussing this physical system will be marked by a $\physicalsystemexample$ to set them apart from the rest of the text.

\subimportlevel{./}{coupled_oscillator}{1}
\subimportlevel{./}{basic_definitions_in_vector_spaces}{1}
\subimportlevel{./}{the_big_picture}{1}
\subimportlevel{./}{linear_transformations_and_their_matrices}{1}
\subimportlevel{./}{generalization_to_complex_vector_spaces}{1}

