\levelstay{Introduction}

This section introduces the main message of the document by dissecting an extremely simple idea in detail.
In linear algebra books and courses, the archetypical vector space is the set of tuples of $n$ numbers, called $\reals^n$.
It's easy to see that this set is indeed a vector space: you can add n-tuples together and multiply them by scalars; for example $(1,4,3) + (0,1,5) = (1,5,8)$.
Of course, in applications, we are not intrinsically interested in the mathematical space $\reals^n$ itself but rather we use elements in $\reals^n$ to represent the objects of interest.
For example, we can use lists of two numbers to represent displacement in a two dimensional plane.
In Figure~\ref{fig:ch.finite_dimensions.sec.big_picture:arrows}\,(a) we show two arrows $\ket{u}$ and $\ket{v}$, each of length 1.
These arrows exist on the page without any further mathematical representation.
If we let $\ket{x}$ represent a unit arrow along the $x$ axis and $\ket{y}$ represent a unit arrow along the $y$ axis, then our two arrows could represented as,
\begin{align}
  \ket{u}
  &= \frac{1}{\sqrt{2}} \left( \ket{x} + \ket{y} \right) \sim \frac{1}{\sqrt{2}} \left[ \begin{array}{c} 1 \\ 1 \end{array} \right] \\
  \ket{v}
  &= \frac{1}{\sqrt{2}} \left( -\ket{x} + \ket{y}\right) \sim \frac{1}{\sqrt{2}} \left[ \begin{array}{c} -1 \\ 1 \end{array} \right]
\end{align}
This representation is useful for several reasons including that it enables finding the arrow we'd get by ``summing'' $\ket{u}$ and $\text{v}$ tip to tail, as shown in Fig.~\ref{fig:ch.finite_dimensions.sec.big_picture:arrows}\,(b).
Adding the tuples together gives
\begin{displaymath}
  \ket{u} + \ket{v} = \frac{1}{\sqrt{2}} \left( \ket{x} + \ket{y} - \ket{x} + \ket{y} \right) = \frac{\ket{y}}{\sqrt{2}}
  \sim
  \begin{bmatrix}
    \sqrt{2} \\
    0
  \end{bmatrix}
\end{displaymath}
which we interpret to mean an arrow of length $\sqrt{2}$ pointing directly upward on the page, which is the correct result of summing the arrows.

We could choose another set basis axes as shown by $\ket{x'}$ and $\ket{y'}$ in Fig.~\ref{fig:ch.finite_dimensions.sec.big_picture:arrows}\,(c).
Then our arrows are represented as
\begin{align}
  \ket{u}
    &= \ket{x'} \sim \left[\begin{array}{c} 1 \\ 0 \end{array} \right] \\
  \ket{v}
    &= \ket{y'} \sim \left[\begin{array}{c} 0 \\ 1 \end{array} \right]
  \, .
\end{align}
The arrows $\ket{u}$ and $\ket{v}$ didn't change but the tuples in $\reals^2$ representing them did change; in fact they became simpler, having only one nonzero entry each instead of two.
That's the guiding principle of this document: the tuples of numbers (or in infinite dimensions, the functions) that we use to represent things depend on the choice of basis.
So, when we encounter tuples of numbers in science or math, it may be useful to consider the underlying objects and consider whether or not there's a basis where the calculation would be simplified or where the numbers representing the underlying objects would be more intuitive.

The rest of this document uses that principle to develop the mathematics behind changing basis in a language that helps draw attention to the fact that the tuples of numbers and matrices involved are all representations of more fundamental underlying objects.
We then derive the Fourier transform as the result of an obvious choice of basis when dealing with differential equations.

\quickfig{\columnwidth}
{arrows.pdf}
{Arrows in a plane. (a) Two arrows, $\ket{u}$ and $\ket{v}$. (b) $\ket{x}$ and $\ket{y}$ (grey arrows) as a basis. Adding $\ket{u}$ and $\ket{v}$ tip-to-tail results in a new arrow (green) pointing straight up, with length $\sqrt{2}$. (c) A different basis in which $\ket{u}$ and $\ket{v}$ are congruent with the basis arrows themselves.}
{fig:ch.finite_dimensions.sec.big_picture:arrows}
