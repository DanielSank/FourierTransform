\section{Properties of the Fourier Transform}

Consider a vector $\ket{f}$ in $L^2$.  We usually specify this vector through knowledge of its components in the $T$ basis, ie. we know the numbers $f(t)$.  We showed that we can consider its components in another basis, $\Omega$ and that these numbers are given by
\begin{displaymath} \int e^{-i\omega t}~f(t)~dt. \end{displaymath}
This has no $t$ dependence because we integrate over $t$, but it does have dependence on $\omega$, which makes sense because the components of $\ket{f}$ along each basis vector in the $\Omega$ basis should, in general, be different.  Since this integral depends on $\omega$ we can consider it to be a function of $\omega$ and we denote it by $\tilde{f}(\omega)$.  In other words
\begin{displaymath} \tilde{f}(\omega) \equiv \int e^{-i\omega t}~f(t)~dt. \end{displaymath}
Now we'll investigate how the components of a vector $\ket{f}$ in the $\Omega$ and $T$ bases are related to one another, or in other words, how the function $\tilde{f}(\omega)$ is related to the function $f(t)$.

%%%subsection Fourier Transform of a Delta Function%%%%%%%%%%%%%%%%%%%%%%%%%%%%%%%%%%%%%%%%%%
\subsection{Fourier Transform of a Delta Function}
One pressing question is the following: what does the vector $\ket{\delta_t}$ look like in the $\Omega$ basis?  This is easy to answer via the following calculation
\begin{displaymath} \tilde{\delta_t} (\omega) \equiv \braket{E_\omega}{\delta_t} \equiv \int e^{-i\omega t'}~\delta_t (t')~dt' = e^{-i\omega t}.\end{displaymath}
So a vector that is represented by a delta function in the time basis is represented by an exponential in the $\Omega$ basis.  Now lets ask the reverse question.  If a vector is represented by an exponential in the time basis, how does it look in the $\Omega$ basis?  A reasonable guess is that an exponential might get Fourier transformed into a delta function, and that's exactly what happens.  Before we show why this is the case, we have to discuss the inverse Fourier transform.

%%%subsection The Inverse Fourier Transform%%%%%%%%%%%%%%%%%%%%%%%%%%%%%%%%%%%%%%%%%%%%%%%%%%%
\subsection{The Inverse Fourier Transform}
We've seen that the Fourier transform turns one functional representation of a vector in $L^2$ into another different functional representation.  How do we reverse this operation?  If we have a function $f(t)$ which gets Fourier transformed into $\tilde{f}(\omega)$ how do we reconstruct the function $f(t)$ from $\tilde{f}(\omega)$?  Let's look at this from our abstract vector point of view.  Asking for the function $f(t)$ is the same as asking for the components of $\ket{f}$ in the $T$ basis.  Of course we can find these components by taking the inner product of $\ket{f}$ with the basis vectors $\ket{\delta_t}$.  Now, we're assuming that we know the components of $\ket{f}$ \emph{in the $\Omega$ basis} so to compute the desired inner product we have to work in the $\Omega$ basis.

\begin{flushleft} \textbf{Exercise} \end{flushleft}
\begin{itemize}\item[1)] Write down the integral for the inner product of $\ket{\delta_t}$ and $\ket{f}$ carried out in the $\Omega$ basis.  Do not read on until you have solved this exercise.  If you cannot solve it go back and make sure you fully understand the material in this chapter, or ask someone for help.
\end{itemize}

Your answer to this exercise should have been
\begin{displaymath} f(t) = \braket{\delta_t}{f} = \int \braket{E_\omega}{\delta_t}^*\braket{E_\omega}{f}~d\omega = \int e^{i\omega t}~\tilde{f}(\omega )~d\omega \end{displaymath}
and this is almost exactly right but it turns out that it's off by a multiplicative factor of $\frac{1}{2\pi}$.  The reason for this is not at all obvious and is actually a result of the fact that when we looked for eigenvectors of the operation $-iD_t$ we chose those represented in the time basis by $E_\omega (t) = e^{i\omega t}$ whereas we could have included any constant multiplicative factor we wanted in the exponent.  The upshot is that to take inner products in the $\Omega$ basis you have to include the factor of $\frac{1}{2\pi}$, ie.
\begin{displaymath} \braket{f}{g} = \frac{1}{2\pi}\int \braket{E_\omega}{f}^*\braket{E_\omega}{g}~d\omega. \end{displaymath}
With this in mind we can see that the inverse Fourier transform is given by
\begin{displaymath}
f(t) \equiv \braket{\delta_t}{f} = \frac{1}{2\pi} \int \braket{E_\omega}{\delta_t}^*\braket{E_\omega}{f}~d\omega
\end{displaymath}
Using the property that $\braket{v}{w}^* = \braket{w}{v}$ we can rewrite this as
\begin{eqnarray*}
f(t) &=& \frac{1}{2\pi} \int \braket{\delta_t}{E_\omega}\braket{E_\omega}{f}~d\omega \\
f(t) &=& \frac{1}{2\pi} \int e^{i\omega t}~\tilde{f}(\omega)~d\omega
\end{eqnarray*}
This is a nice result: the inverse Fourier transform looks almost exactly like the Fourier transform.  The only differences are the factor of $\frac{1}{2\pi}$ and the minus sign in the exponent of the exponential function.  To sum this up, we have the following relations between a function and its Fourier transform
\begin{displaymath}
\framebox{
$\tilde{f}(\omega) = \int e^{-i\omega t} f(t)~dt$ ~ and ~
$f(t) = \frac{1}{2\pi} \int e^{i\omega t} \tilde{f}(\omega)~d\omega$
}
\end{displaymath}
From our abstract vector approach it follows that the inverse Fourier transform of the Fourier transform of a function returns the original function,
\begin{displaymath} f(t) \stackrel{\textrm{FT}}{\longrightarrow} \tilde{f}(\omega ) \stackrel{\textrm{inverse FT}}{\longrightarrow} f(t). \end{displaymath}

%%%subsection Fourier Transform of an Exponential%%%%%%%%%%%%%%%%%%%%%%%%%%%%%%%%%%%%%%%%%%%%%
\subsection{Fourier Transform of an Exponential}
Suppose that we want to know the components of the vector $\ket{E_\omega}$ in the $\Omega$ basis.  We begin by just writing down the necessary inner product.  Let's calculate the component of the vector $\ket{E_\omega}$ along the $\Omega$ basis vector $\ket{E_{\omega '}}$.  We have to do the integral in the $T$ basis because we \emph{don't know} the components of $\ket{E_\omega}$ in the $\Omega$ basis.
\begin{eqnarray*}
\braket{E_{\omega '}}{E_\omega} &=& \int \braket{\delta_t}{E_{\omega '}}^* \braket{\delta_t}{E_\omega}~dt\\
\tilde{E}_{\omega}(\omega ') &=& \int (e^{i\omega ' t})^*~e^{i\omega t}~dt\\
\tilde{E}_{\omega}(\omega ') &=& \int e^{-i\omega ' t}~e^{i\omega t}~dt\\
\tilde{E}_\omega (\omega ') &=& \int e^{i(\omega - \omega ')t}~dt
\end{eqnarray*}
There are a few ways of going about doing this integral.  One way is to use the methods of contour integration from complex analysis, but we're not assuming that you are familiar with these methods so we won't use them.\footnote{Contour integration is extraordinarily useful for both raw computational power and for conceptual understanding of many areas of mathematics. Any first level math course in complex analysis will cover contour integration.}  Another trick is to restrict the limits of integration to some finite size, say $(-D,+D)$ and then take the limit that $D\rightarrow \infty$ at the end.  This is a good trick but it would involve discussion of some new mathematical objects which are not appropriate for this discussion.  We'll use a third method which takes advantage of our abstract understanding of vectors.

A great teacher once told me ``never do a calculation if you don't already know the answer.'' He meant that you should always observe the properties of a calculation before you set out do nail it down rigorously.  Let's take his advice for a moment.

\begin{flushleft}\textbf{Exercise}\end{flushleft}
\begin{itemize}
\item[2)] Make a list of facts that you can deduce about the integral $\int e^{i(\omega - \omega ')t}~dt$ without doing any calculations. Take this seriously and don't go on until you've come up with something nontrivial.
\end{itemize}

One of the most obvious properties of the integral is that it is infinity if $\omega = \omega '$. You should have written that down. For $\omega \neq \omega'$ the integrand is oscillatory. In that case it doesn't look like it will converge to anything, it just oscillates about zero. Hmmm, infinity for one case and sort of zero in every other case. This smells like a delta function. Now that we have some idea of what the answer \emph{might} be we can check it. 

We understand the vectors $\ket{E_\omega}$ to be the basis vectors from the basis $\Omega$.  Therefore, if we know the components of another arbitrary vector $f$ in the $\Omega$ basis, then we should

\begin{flushleft} $\clubsuit$ \end{flushleft}
Equation (\ref{eq:AbstractForDamped}) says
\begin{displaymath} Z\ket{\Psi} = \ket{G}. \end{displaymath}
The natural thing to do now is to project the equation into the $\omega$ basis as follows
\begin{eqnarray*}
\braket{E_{\omega}}{Z|\Psi} &=& \braket{E_{\omega}}{G}\\
\braket{E_{\omega}}{(D_t^2 + \beta D_t + \omega _0^2)|\Psi} &=& \braket{E_{\omega}}{G}.
\end{eqnarray*}
Now we use the fact, which we showed in the previous section, that $\braket{E_\omega}{D_tf}=i\omega \braket{E_\omega}{f}$,
\begin{eqnarray*}
(-\omega^2 + i\frac{\gamma}{m}\omega + \omega_0^2) \braket{E_{\omega}}{\Psi}&=&\braket{E_{\omega}}{G}\\
\braket{E_{\omega}}{\Psi}&=&\frac{\braket{E_{\omega}}{G}}{(-\omega^2 + i\frac{\gamma}{m}\omega + \omega_0^2)}.
\end{eqnarray*}
Ok, that's pretty good, now we know the components of $\ket{\Psi}$ in the $\omega$ basis.  Of course, we're really interested in $\Psi(t)$ not $\Psi(\omega)$ so we have a little more work to do.
\begin{eqnarray*}
\braket{\delta_t}{\Psi} &=& \int _{\omega} \braket{E_{\omega}}{\delta_t}^*~\braket{E_{\omega}}{\Psi}~d\omega\\
\Psi(t) = \braket{\delta_t}{\Psi} &=& \int _{\omega} \frac{e^{i\omega t} \braket{E_{\omega}}{G}}{(-\omega^2 + i\frac{\gamma}{m}\omega + \omega_0^2 )}~d\omega.
\end{eqnarray*}
This is, in principle, the solution to the problem.  The only thing left to do in practice is to figure out what $\braket{E_{\omega}}{G}$ is and then evaluate the integral over $\omega$.

For the sake of completeness and to show you how to do these calculations, let's pick a particular form for $\ket{G}$ and take the problem to the very end.  A common force that we encounter in real life is something like $F_a(t) = F_0\cos(\omega_F t)$.  By this point you should understand that by giving the force as a function of time we have specified its components in the time basis.  In other words, we've given the numbers $\braket{\delta_t}{F_a} = F_a(t) = F_0\cos(\omega_F t)$.  What we need to solve our integral are the numbers $\braket{E_{\omega}}{F_a}$, so we calculate,
\begin{eqnarray*} \braket{E_{\omega}}{F_a} &=& \int _t \braket{\delta_t}{E_{\omega}}^* \braket{\delta_t}{F_a}~dt \\
&=& \int _t e^{-i\omega t} F_a(t)~dt \\
&=& \int _t e^{-i\omega t} F_0 \cos (\omega_F t)~dt \\
\braket{E_{\omega}}{F_a} &=& \frac{1}{2} (\delta_{\omega_F} + \delta_{-\omega_F}). \end{eqnarray*}
An exercise will explain why the last line follows from the one before it.  Now we can substitute this into our previous expression for $\Psi(t)$.
\begin{eqnarray*}
\Psi(t) &=& \int _{\omega} e^{i\omega t} \frac{1}{m}\frac{\braket{E_{\omega}}{F_a}}{(-\omega^2 + i\frac{\gamma}{m}\omega + \omega_0^2)}~d\omega \\
\Psi(t) &=& \int _{\omega} e^{i\omega t} \frac{1}{m}\frac{\frac{1}{2}(\delta_{\omega_F} + \delta_{-\omega_F})}{(-\omega^2 + i\frac{\gamma}{m}\omega + \omega_0^2)}~d\omega \\
\Psi(t) &=& \frac{1}{2\pi}\frac{1}{2m} (\frac{e^{i\omega_Ft}}{-\omega_F^2 + i\frac{\gamma}{m}\omega_F + \omega_0^2} + \frac{e^{-i\omega_Ft}}{-\omega_F^2 - i\frac{\gamma}{m}\omega_F + \omega_0^2})~d\omega \\
\Psi(t) &=& \frac{1}{4\pi m} \textrm{Re} (\frac{e^{i\omega_Ft}}{-\omega_F^2 + i\frac{\gamma}{m}\omega_F + \omega_0^2})
\end{eqnarray*}
This is the solution to the original problem.
\begin{flushright} $\clubsuit$ \end{flushright}